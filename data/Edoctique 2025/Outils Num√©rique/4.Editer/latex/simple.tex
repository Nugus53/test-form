\documentclass[12pt]{article}
\usepackage[utf8]{inputenc} 

\title{Mon Premier Document LaTeX}
\author{Votre Nom}
\date{\today}

\begin{document}

\maketitle % Crée le titre, l'auteur et la date

\section{Introduction}
Bienvenue dans ce document LaTeX de base. Vous allez apprendre à utiliser certaines des commandes les plus courantes.

\subsection{Qu'est-ce que LaTeX ?}
LaTeX est un système de composition de documents qui permet de créer des documents scientifiques de manière professionnelle. 

\section{Mise en Forme du Texte}
Voici quelques commandes de base pour la mise en forme du texte :
\begin{itemize}
    \item \textbf{Gras} : \texttt{\textbackslash textbf\{texte\}} — Exemple : \textbf{texte en gras}.
    \item \textit{Italique} : \texttt{\textbackslash textit\{texte\}} — Exemple : \textit{texte en italique}.
    \item \underline{Souligné} : \texttt{\textbackslash underline\{texte\}} — Exemple : \underline{texte souligné}.
    \item \texttt{Monospace} : \texttt{\textbackslash texttt\{texte\}} — Exemple : \texttt{texte en monospace}.
\end{itemize}
\newpage
\section{Listes}
Voici un exemple de liste à puces :
\begin{itemize}
    \item Premier élément
    \item Deuxième élément
    \item Troisième élément
\end{itemize}

Et un exemple de liste numérotée :
\begin{enumerate}
    \item Premier point
    \item Deuxième point
    \item Troisième point
\end{enumerate}
\end{document}