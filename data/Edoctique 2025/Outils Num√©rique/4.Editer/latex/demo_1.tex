 \documentclass{book}
\usepackage{ekdosis}
\usepackage{xltabular}
\SetSeparator{; }
\SetSubseparator{,}
\DeclareWitness{Lag}{Lag}{Liber interpretationis hebraicorum nominum, Paul de Lagarde, 1959, p. 59-161 (CCSL 72)}
\DeclareWitness{H}{H}{Bamberg, Staatsbibliothek, Msc. Bibl. 154, IX\textsuperscript{e} siècle, f. 101v-178v}
\DeclareWitness{C}{C}{Colmar, Bibliothèque municipale, 33 (41), IX\textsuperscript{e} siècle, f. 81v-118r}
\DeclareWitness{K}{K}{Karlsruhe, Badische Landesbibliothek, Aug. Perg. CCXVIII, IX-X\textsuperscript{e} siècle, f. 82-127v}
\DeclareWitness{L}{L}{Laon, Bibliothèque municipale 24, IX\textsuperscript{e} siècle, f. 1v-67v}
\DeclareWitness{T}{T}{Monte Cassino, Archivio dell’Abbazia, 316 F, IX\textsuperscript{e} siècle, f. 48-142}
\DeclareWitness{O}{O}{Oxford, Bodleian Library, Marshall 19, IX\textsuperscript{e} siècle, f. 1v-42v}
\DeclareWitness{Σ}{Σ}{Sankt Gallen, Stiftsbibliothek, Sang. 130, VIII-IX\textsuperscript{e} siècle, f. 267-355}

\begin{document}

\small
\begin{xltabular}{1\linewidth}{lX}
\caption*{\textbf{Conspectus siglorum}}\\

\SigLine{C}  \\
\SigLine{H} \\
\SigLine{K} \\
\SigLine{L} \\
\SigLine{T} \\
\SigLine{O} \\
\SigLine{Σ} \\
\SigLine{Lag} 
\end{xltabular}

\section{Lettre I}
 \begin{alignment}[tcols=2,
lcols=2,
texts=latin[xml:lang="la"];
french[xml:lang="fr"],
apparatus=latin,
lineation=page]


\begin{latin}
\noident\app{
\lem[wit={L,O,T}]{Iatham}
\rdg[wit=C]{Ieththam}
\rdg[wit=H]{Ietham}
\rdg[wit=K]{Iotam}
\rdg[wit={Lag}]{Iotham}
\rdg[wit={Σ}]{Iathan}
}: Domini
\app{
\lem[wit={C,H,K,L,Lag,O,T}]{consummatio}
\rdg[wit=Σ]{consumatio}
}
\app{
\lem[wit=C,H,K,L,Lag,Σ]{siue}
\rdg[wit=O,T]{uel}
}
perfectio. \\
Iezrahel: Semen dei\\
\app{
\lem[wit={C,H,K,L,Lag,O,T}]{Iarib}
\rdg[wit=Σ]{Iarith}
}: \app{
\lem[wit={C,H,K,L,Lag,T,Σ}]{Diiudicans}
\rdg[wit=O]{Deiudicans}
} uel 
\app{
\lem[wit={C,H,K,L,Lag,O,Σ}, alt=\emph{om.}]{}
\rdg[wit={T}]{de}
}ulciscens \app{
\lem[wit={C,H,K,L,Lag,O,T}, alt=\emph{om.}]{}
\rdg[wit=Σ]{superior}
}
\end{latin}

\begin{french}
\noident Iatham: L'accomplissement ou la perfection du Seigneur\\
Iezrahel: La semence de Dieu\\
Iarib: Jugeant ou vengeant\\
\end{french}
\end{alignment}

\end{document}

 \end{document}